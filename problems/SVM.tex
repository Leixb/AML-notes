%! TEX root = ../000-main.tex
% 2022-10-24
\subsection{SVM exercises}

\begin{exercise}[breakable]{}{}
	Consider a hard-margin SVM. We know that for all Svs $x_i$:
	\begin{equation}
		\label{eq:svm:margin}
		\widehat\omega^T x^i + \widehat{b} = y^i
	\end{equation}

	\begin{enumerate}
		\item Obtain an expression for $\widehat{b}$ in therms of the
		      dual variable, $\widehat{\alpha}_i$.
		      \begin{align*}
			      \widehat\omega & = \sum_{i=1}^n \widehat\alpha_i y^i x^i                            \\
			      \widehat{b}    & = y^i - \left( \sum_{j=1}^n \widehat\alpha_j y^j x^j \right)^T x^i \\
			                     & = y^i - \sum_{j=1}^n \widehat\alpha_j y^j \left(x^j\right)^T x^i
		      \end{align*}

		      \begin{note}
			      In practice, we use more than one support vector:
			      \begin{equation*}
				      %\doublehat
				      \widehat b = \frac{1}{\left|S\right|} \sum_{i \in S}
				      \left(
				      y^i - \sum_{j=1}^n \widehat\alpha_j y^j \left(x^j\right)^T x^i
				      \right)
			      \end{equation*}
		      \end{note}

		\item Derive a simple expression for the margin of the dual.

      Using the expression for $\widehat b$ from the previous part, we can
      write the margin as:
		      \begin{align*}
            \widehat{b}   & = y^i - \sum_{j=1}^n \widehat\alpha_j y^j \left(x^j\right)^T x^i \label{eqstep:1}\tag{b} \\
            m &= \frac{2}{\lVert \widehat\omega \rVert} \tag{target}\\
            \sum_{i=1}^n \widehat\alpha_i y^i &= 0 \\
            \span\text{We take \cref{eqstep:1} and multiply by } (\widehat\alpha_iy^i): \\
            \widehat\alpha_iy^i \widehat b &= \widehat\alpha_iy^i y^i - \sum_{j=1}^n \widehat\alpha_iy^i \widehat\alpha_j y^j \left(x^j\right)^T x^i \\
            \cancelto{0}{\widehat\alpha_iy^i \widehat b} &=
            \widehat\alpha_i \cancelto{1}{y^i y^i} - \sum_{j=1}^n \widehat\alpha_iy^i \widehat\alpha_j y^j \left(x^j\right)^T x^i \\
            0 &= \sum_{i=1}^n \widehat\alpha_i
            - \sum_{i=1}^n \sum_{j=1}^n \widehat\alpha_i \widehat\alpha_j y^i y^j \left(x^j\right)^T x^i \\
            \sum_{i=1}^n \widehat\alpha_i &= \sum_{i=1}^n \sum_{j=1}^n \widehat\alpha_i \widehat\alpha_j y^i y^j \left(x^j\right)^T x^i = \left(\widehat\omega\right)^T \widehat\omega \\
            \sum_{i=1}^n \widehat\alpha_i &= \lVert \widehat\alpha \rVert_1 \\
            \widehat\alpha_i \geq 0 &\implies \lVert \widehat\alpha \rVert_1 = \lVert \widehat\omega \rVert^2 \\
            m &= \lVert \widehat\alpha \rVert_1^{-1/2}
		      \end{align*}
	\end{enumerate}
\end{exercise}

\begin{exercise}{}{}
	Consider again a hard-margin SVM in the dual form. Express
	the optimization problem as a minimization over a
	recognizable QP (quadratic problem), in vector-matrix form.
\end{exercise}

\begin{exercise}{}{}
  Consider this hypothesis space in $\mathds{R}$:
  \begin{equation*}
    \mathcal{F}_\text{SVM} := \left\{
      x \to \text{sign}\left( \sin (\alpha x)\right),\; \alpha \in \mathds{R}
    \right\},\; x \in \mathds{R}
  \end{equation*}

  Calculate the VC-dimension of $\mathcal{F}_\text{SVM}$.

  \tcblower

  The VC-dimension is intrinsically related to the number
  of parameters.

  You choose $n \geq 2$

  \begin{align*}
    x^i := 10^{-i},\; i \in \{1, \ldots, n\} \\
    \alpha_j := \pi \left[
      1 + \sum_{i=1}^n \left( \frac{1}{2}(1 + ?)1-10^i \right)^i
    \right] \\
    y = (-1, +1, -1, +1, -1, +1, -1, +1, -1, -1,  +1, +1, -1, +1, +1)
  \end{align*}

\end{exercise}
