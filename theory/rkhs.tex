%! TEX root = ../000-main.tex
\chapter{Learning in RKHSs}

\begin{definition}{Hilbert space}{hilber-space}
	\begin{itemize}
		\item Consider a space $G$ of functions $f: \mathcal{X} \to \mathbb{R}$,
		      where $\mathcal{X}$ is the data space.

		\item Define an \iemph{inner product} (ip) $\langle f, g \rangle, \; f, g \in G$ by
		\item $\{f_n\}_{n \in \mathds{N}}$ converges to $f$ iff
		      $\forall \epsilon > 0, \exists N \in \mathds{N} \text{ s.t. } \forall n \geq N, \; \lVert f_n - f \rVert < \epsilon$.

		\item \iemph{Cauchy sequence}
		      $\{f_n\}_{n \in \mathds{N}}$ is Cauchy when:
		      $\forall \epsilon > 0, \; \exists N \in \mathds{N} \text{ s.t. } \forall n, m \geq N, \; \lVert f_n - f_m \rVert < \epsilon$.
	\end{itemize}

	A space in which all Cauchy sequences converge to an element of the space is
	called \emph{complete}.

	A complete vector space in which we have
	a norm ($\lVert \cdot \rVert$) is called a \iemph{Banach space}.

	\begin{marker}
		A \iemph{Hilbert space} (HS) is a Banach space where the norm
		comes from an inner product.
	\end{marker}

\end{definition}

\begin{question}{Homework}{}
	Is there an implication between convergence and Cauchy convergence?
\end{question}

\begin{example}{Hilbert spaces}{}
	\begin{itemize}
		\item $\mathds{R}^d, \; d \in \mathds{N}$
		\item The space of square-integrable functions
	\end{itemize}
\end{example}

\begin{definition}{RKHSs}{RKHSs}
	$\mathcal{X}$ data space ($\mathds{R}^d$)

	$\mathcal{H}$ of functions $f: \mathcal{X} \to \mathds{R}$

	A \iemph{Reproducing Kernel Hilbert space} (RKHS) is a Hilbert space
	of functions in which all linear evaluation functionals are continuous.

	\begin{align*}
		[x](\cdot) : \mathcal{H} & \to \mathds{R}                                     \\
		f                        & \mapsto [x](f) = f(x)                              \\
		\span
		\text{ is continuous } \forall f \in \mathcal{H} \; \forall x \in \mathcal{X} \\
		\iff \text{ all functionals are bounded }
	\end{align*}
\end{definition}

\begin{theorem}{}{}
	In a RKHS $\mathcal{H}$, $\forall x \in \mathcal{X}$, there
	exists an element $\kappa_x(\cdot) \in \mathcal{H}$, which
	``represents'' the evaluation functional $[x](\cdot)$
	in this sense:
	\begin{align*}
		\langle \kappa_x(\cdot), f \rangle_{\mathcal{H}} = f(x) \forall f \in \mathcal{H}
	\end{align*}
\end{theorem}

\begin{marker}
	$\kappa_x(\cdot)$ is a function
\end{marker}

\begin{definition}{Reproducing kernel}{}
	The symmetric function
	\begin{equation*}
		\kappa(x, x') := \kappa_x(x'),\; \forall x, x' \in \mathcal{X}
	\end{equation*}
	is called the \iemph{reproducing kernel} of the space (rk).
\end{definition}

\begin{definition}{Positive semi-definite functions}{psd}
	Consider a data space $\mathcal{X}$. A symmetric bivariate function
	$\kappa: \mathcal{X} \times \mathcal{X} \to \mathds{R}$ is
	\iemph{positive semi-definite} (PSD) when
	$\forall n \in \mathds{N}^+, \forall \{x^1,\, x^2,\,\dots, x^n\},
		x^i \in \mathcal{X}$.

	\begin{align*}
		\mathlarger{\sum_{i=1}^n} \sum_{j=1}^n  a_i a_j \kappa(x^i, x^j) \geq 0,\quad
		\forall a=\begin{pmatrix} a_1 \\ \vdots \\ a_n \end{pmatrix}, a_i \in \mathds{R}
	\end{align*}

	\tcblower

	(alternative) If we form the following matrix:
	\begin{align*}
		\mathlarger{K}_{ij} := \kappa(x^i, x^j), \text{ then } a^TKa \geq 0
	\end{align*}
\end{definition}

\begin{example}{}{}
	\begin{align*}
		\mathcal{X} & = \{1, 2, 3, \dots, N \} \rightarrow{}
		\text{ In this case } \kappa \text{ is empty } K     \\
		\mathcal{X} & = [a,b] \subset \mathds{R}             \\
		\mathcal{X} & = \mathds{R}^d                         \\
	\end{align*}
\end{example}

\begin{theorem}[parbox=false]{Moore-Aronszajn '50s}{}
	For every PSD funciton $\kappa: \mathcal{X} \times \mathcal{X} \to \mathds{R}$, there
	is a \emph{unique} RKHS $\mathcal{H}_\kappa$ for which $\kappa$ is the reproducing kernel.

	\tcblower

	More precisely,

	Define $\kappa_x(\cdot) := \kappa(x, \cdot)$; then
	$\forall f \in \mathcal{H}_\kappa,
		\langle \kappa_x, f \rangle = f(x), \forall x \in \mathcal{X}$.

	The converse is also true:
	If we depart from a RKHS $\mathcal{H}_{\kappa}$ having
	$\kappa$ as reproducing kernel, then $\kappa$ is PSD function.
\end{theorem}

\section{Construction of the RKHS}

Depart from a PSD function $\kappa: \mathcal{X} \times \mathcal{X} \to \mathds{R}$.
And a dataset $\mathcal{D} = \{x^1, \dots, x^n\}$.

\begin{itemize}
	\item $\mathcal{H}_\kappa := \emptyset$.
	\item $\forall x \in \mathcal{X}$, define $\kappa_x(\cdot) := \kappa(x, \cdot)$.
	      and $\forall x \in \mathcal{X}$, add $\kappa_x$ to $\mathcal{H}_\kappa$.
	\item Add to $\mathcal{H}_\kappa$ its own \iemph{span}:
	      \begin{equation*}
		      \mathcal{H}_\kappa := \mathcal{H}_\kappa \cup \{
		      f(\cdot) = \sum_{i=1}^n a_i \kappa_{x^i}(\cdot) \mid
		      \forall x \in \mathcal{X},
		      \forall n \in \mathds{N}^+,
		      \forall a_i \in \mathds{R},
		      i=1,\dots,n
		      \}
	      \end{equation*}
	\item We now define the inner product:
	      \begin{align*}
          \langle f, g \rangle_{\mathcal{H}_\kappa} &:=
		      \left\langle
            \sum_{i=1}^n a_i \kappa_{x^i}(\cdot), \sum_{j=1}^{n'} a'_j \kappa_{x'^j}(\cdot)
		      \right\rangle \\
                                                    &= \mathlarger{\sum_{i=1}^n} \sum_{j=1}^{n'} a_i a'_j \kappa({x^i}, {x'^j})
	      \end{align*}
      \item Let's show that $\kappa$ is the r.k of $\mathcal{H}_\kappa$
        \begin{align*}
          \forall x &\in \mathcal{X} \\
          \langle \kappa_x, f \rangle_{\mathcal{H}_\kappa} &=
          \left\langle \kappa_x, \sum_{i=1}^n a_i \kappa_{x^i} \right\rangle_{\mathcal{H}_\kappa} = \\
          \sum_{i=1}^n a_i \kappa(x, x^i) &= f(x)
        \end{align*}
      \item \iemph{Completeness}: We take all limits of Cauchy sequences
        and add them to the space.
\end{itemize}

\begin{definition}{Span of a set}{}
	\iemph{Span of a set} is the set of all linear combinations of the elements of the set.
\end{definition}

\begin{theorem}{Representer theorem}{}
\end{theorem}
